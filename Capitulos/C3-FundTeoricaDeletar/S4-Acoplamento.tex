\documentclass[_ArquivoPrincipal.tex]{subfiles}

\begin{document}

%==================================================================================================
\section{Acoplamento Fluido-Estrutura} \label{AFE}
%==================================================================================================

No intuito de realizar o acoplamento entre o fluido e a estrutura, denota-se $\Omega_F$ o domínio do fluido, $\Omega_E$ o domínio da estrutura, $\Omega_{IFE}=\Omega_S\cup\Omega_E$ o domínio do problema e $\Gamma_{IFE}=\Omega_S\cap\Omega_E$ a interface de IFE.

Nesse sentido \citeonline{richter2017fluid} apontam três condições auxiliares para que haja a correta interação: a Condição Cinemática, que diz respeito à movimentação dos domínios analisados, devendo ser compatíveis em $\Gamma_{IFE}$, ou seja, a componente normal ao movimento deve ser igual em ambos os meios, assim como a componente tangencial, em caso de aderência; a Condição Dinâmica, que aponta a continuidade das forças internas, observadas no tensor de tensões de Cauchy; e a Condição Geométrica, que exige a necessidade de ambos os domínios coincidirem em $\Gamma_{IFE}$, não havendo sobreposições nem formação de vazios.

A determinação do comportamento dos diferentes meios que compõem o problema de IFE pode ser obtido de diferentes formas, sendo observados o acoplamento monolítico e o acoplamento particionado, sendo o último ainda subdividido em particionado fraco e forte.

Nos casos estudados para CFD e CSD, é comumente empregado o Método de Newton-Raphson para o cálculo dos parâmetros de interesse, o que leva à definição de uma matriz tangentes ($\BB{H}$), a qual pode ser obtida por \cite{bazilevs2013computational,sanches2022metodos}:

\begin{equation}
    H_{ij}^k=\Dder{\script{G}}{\beta_i^k}{\alpha_j^k}\text{,}
\end{equation}

\noindent sendo $\script{G}$ a soma de todas as equações diferenciais do problema em sua forma fraca, $\beta_i^k$ o vetor com todos os parâmetros nodais das funções testes e $\alpha_j^k$ o vetor com todos os parâmetros nodais do problema. Assim, obtém-se a correção dos parâmetros nodais ($\Delta\BB{\alpha}^k$) por meio da solução do sistema:

\begin{equation}
    \BB{H}^k\cdot\Delta\BB{\alpha}^k=-\BB{h}^k\text{,}
\end{equation}

\noindent em que $\BB{h}=\partial\script{G}/\partial\beta_i^k$ é o vetor resíduo.

Expandindo a matriz em submatrizes a fim de visualizar a contribuição de cda parcela no sistema global tem-se:

\begin{equation}
    \begin{bmatrix}
        \BB{H}_{11}^k & \BB{H}_{12}^k & \BB{H}_{13}^k \\
        \BB{H}_{21}^k & \BB{H}_{22}^k & \BB{H}_{23}^k \\
        \BB{H}_{31}^k & \BB{H}_{32}^k & \BB{H}_{33}^k
    \end{bmatrix}\cdot\begin{bmatrix}
        \Delta\BB{\alpha}_1^k \\\Delta\BB{\alpha}_2^k\\\Delta\BB{\alpha}_3^k
    \end{bmatrix}=-\begin{bmatrix}
        \BB{h}_1^k \\\BB{h}_2^k\\\BB{h}_3^k
    \end{bmatrix}\text{,}
\end{equation}

\noindent sendo os subíndices $1$, $2$ e $3$ referentes às formulações do fluido, da estrutura e da malha, respectivamente.

Já o acoplamento particionado trata o sistema de forma a eliminar os termos cruzados da matriz tangente, resultando em blocos de sistema que podem ser resolvidos independentemente:

\begin{equation}
    \begin{bmatrix}
        \BB{H}_{11}^k & 0             & 0             \\
        0             & \BB{H}_{22}^k & 0             \\
        0             & 0             & \BB{H}_{33}^k
    \end{bmatrix}\cdot\begin{bmatrix}
        \Delta\BB{\alpha}_1^k \\\Delta\BB{\alpha}_2^k\\\Delta\BB{\alpha}_3^k
    \end{bmatrix}=-\begin{bmatrix}
        \BB{h}_1^k \\\BB{h}_2^k\\\BB{h}_3^k
    \end{bmatrix}\text{.}
\end{equation}

Para aprimorar os resultados, pode-se atualizar os valores calculados em um bloco para o cálculo do próximo, sendo obtido pelo procedimento apresentado no pseudocódigo \ref{alg:AcoplamentoParticionado}:

\begin{algorithm}[h]
    \caption{Cálculo dos parâmetros nodais}
    \label{alg:AcoplamentoParticionado}
    Resolver o sistema: $\BB{H}_{11}^k(\BB{\alpha}_1^k,\BB{\alpha}_2^k,\BB{\alpha}_3^k)\cdot\Delta\BB{\alpha}_1^k=-\BB{h}_1^k(\BB{\alpha}_1^k,\BB{\alpha}_2^k,\BB{\alpha}_3^k)$\;
    Atualizar parâmetros: $\BB{\alpha}_1^{k+1}\gets\BB{\alpha}_1^k+\Delta\BB{\alpha}_1^k$\;
    Resolver o sistema: $\BB{H}_{22}^k(\BB{\alpha}_1^{k+1},\BB{\alpha}_2^k,\BB{\alpha}_3^k)\cdot\Delta\BB{\alpha}_2^k=-\BB{h}_2^k(\BB{\alpha}_1^{k+1},\BB{\alpha}_2^k,\BB{\alpha}_3^k)$\;
    Atualizar parâmetros: $\BB{\alpha}_2^{k+1}\gets\BB{\alpha}_2^k+\Delta\BB{\alpha}_2^k$\;
    Resolver o sistema: $\BB{H}_{33}^k(\BB{\alpha}_1^{k+1},\BB{\alpha}_2^{k+1},\BB{\alpha}_3^k)\cdot\Delta\BB{\alpha}_3^k=-\BB{h}_3^k(\BB{\alpha}_1^{k+1},\BB{\alpha}_2^{k+1},\BB{\alpha}_3^k)$\;
    Atualizar parâmetros: $\BB{\alpha}_3^{k+1}\gets\BB{\alpha}_3^k+\Delta\BB{\alpha}_3^k$\;
\end{algorithm}

Nesse ponto o acoplamento fraco e forte se diferenciam no sentido de que o acoplamento fraco despreza as etapas de correções das variáveis, ou seja, o valor dessas variáveis em um passo de tempo é unicamente dependente dos valores do passo anterior, caracterizando, assim, como um método explícito.

\citeonline{sanches2022metodos} sugere um procedimento refinado para o cálculo dos parâmetros, em que o problema do fluido depende das variáveis do sólido e da malha em um instante $t$ e das variáveis do fluido em um instante $t+1$, já o problema do sólido depende do valor das variáveis do fluido e do sólido em um tempo $t+1$ e, por fim, o problema da malha depende das variáveis do sólido e da malha em um instante $t+1$. Assim, o cálculo é realizado segundo o Algoritmo \ref{alg:PartFraco}, no qual $k_f$, $k_s$ e $k_m$ são as iterações das soluções dos problemas de fluido, estrutura e malha, respectivamente. Ainda é observado que a movimentação da malha segue um problema linear, sendo seu resultado obtido diretamente em um iteração.

\begin{algorithm}[h]
    \caption{Cálculo dos parâmetros nodais em um acoplamento particionado fraco}
    \label{alg:PartFraco}
    Resolver o sistema: $\BB{H}_{11}((\BB{\alpha}_1^{k_f})^{t+1},(\BB{\alpha}_2)^t,(\BB{\alpha}_3)^t)\cdot\Delta(\BB{\alpha}_1^{k_f})^{t+1}=-\BB{h}_1((\BB{\alpha}_1^{k_f})^{t+1},(\BB{\alpha}_2)^t,(\BB{\alpha}_3)^t)$\;
    Atualizar parâmetros: $(\BB{\alpha}_1^{k_f+1})^{t+1}\gets(\BB{\alpha}_1^{k_f})^{t+1}+(\Delta\BB{\alpha}_1^{k_f})^{t+1}$\;
    Resolver o sistema: $\BB{H}_{22}((\BB{\alpha}_1)^{t+1},(\BB{\alpha}_2^{k_s})^{t+1})\cdot\Delta(\BB{\alpha}_2^{k_s})^{t+1}=-\BB{h}_2((\BB{\alpha}_1)^{t+1},(\BB{\alpha}_2^{k_s})^t)$\;
    Atualizar parâmetros: $(\BB{\alpha}_2^{k_s+1})^{t+1}\gets(\BB{\alpha}_2^{k_s})^{t+1}+(\Delta\BB{\alpha}_2^{k_s})^{t+1}$\;
    Resolver o sistema: $\BB{H}_{3}((\BB{\alpha}_3^{k_m})^{t+1},(\BB{\alpha}_2)^t)\cdot\Delta(\BB{\alpha}_3^{k_s})^{t+1}=-\BB{h}_3((\BB{\alpha}_3^{k_m})^{t+1},(\BB{\alpha}_2)^t)$\;
    Atualizar parâmetros: $(\BB{\alpha}_3^{k_m+1})^{t+1}\gets(\BB{\alpha}_3^{k_m})^{t+1}+(\Delta\BB{\alpha}_3^{k_m})^{t+1}$\;
\end{algorithm}

Por dispensar etapas de correção, esse método possui um custo computacional menor em relação ao particionado forte, no entanto exige a adoção de passos de tempo pequenos, uma vez que os erros inerentes à variação temporal se tornam mais proeminentes em casos com passos de tempo elevados.

Por sua vez, o acoplamento particionado forte realiza etapas de correção dentro do processo iterativo, de forma a corrigir eventuais erros provenientes da solução de cada subproblema.

\textcolor{red}{Comentar sobre instabilidade devido à massa adicionada.}

\end{document}