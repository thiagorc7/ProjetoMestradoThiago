% Introdução

\documentclass[_ArquivoPrincipal.tex]{subfiles}

\begin{document}

\chapter{Álgebra tensorial}\label{ch:algebra}

\section{Contrações}

Quando a forma indicial utiliza apenas um índice mudo, a operação é chamada de
	{contração simples} (denotada por $\cdot$). Quando utilizam-se dois
índices
mudos, essa é chamada de {contração dupla} (denotada por $:$). Alguns exemplos
de contrações são

\begin{align}
	a = \mathbf{b}\cdot\mathbf{c}                       & \iff
	a=b_{i}c_i  ,                                                   \\
	\mathbf{C}=\mathbf{A}\cdot\mathbf{B}                & \iff
	{C}_{ij}={A}_{ik}{B}_{kj} , \label{eq:cont2x2}                  \\
	a=\mathbf{B}:\mathbf{C}                             & \iff
	a=\mathbf{B}_{ij}\mathbf{C}_{ij} ,                              \\
	\mathbf{a}=\mathbf{B}\cdot\mathbf{c}                & \iff
	a_i=B_{ij}c_j ,                                                 \\
	\mathbf{A} = \boldsymbol{\mathfrak{B}}:\mathbf{C}   & \iff
	A_{ij}=\mathfrak{B}_{ijkl}C_{kl} ,                              \\
	a = \mathbf{B}:\boldsymbol{\mathfrak{C}}:\mathbf{D} & \iff  a =
	B_{ij}:{\mathfrak{C}}_{ijkl}:{D}_{kl} \text{ e}                 \\
	\mathfrak{C}=\mathfrak{A}:\mathfrak{B}              & \iff
	\mathfrak{C}_{ijkl}=\mathfrak{A}_{ijmn}\mathfrak{B}_{mnkl}
	.\label{eq:cont4x4}
	% \\
	%\mathbf{A}=\mathbf{B}:\mathfrak{C} &\iff \mathbf{A}_{ij}=\mathbf{B}_{kl}\mathfrak{C}_{klij}
\end{align}

\section{Produtos tensoriais}

A operação de produto tensorial (denotada por $\otimes$), em geral, não utiliza
índices mudos. Alguns exemplos são

\begin{align}
	\mathbf{C}=\mathbf{a}\otimes\mathbf{b}                 & \iff
	{C}_{ij}={a}_{i}{b}_{j} \text{	e}
	\\
	\boldsymbol{\mathfrak{C}}=\mathbf{A}\otimes \mathbf{B} & \iff
	\mathfrak{C}_{ijkl}={A}_{ij}{B}_{kl} \text{.}
\end{align}

\section{Tensores identidade}

Define-se os tensores identidade de segunda e quarta ordem, respectivamente,
por

\begin{gather}
	I_{ij}=\delta_{ij} \text{ e} \label{eq:Id2} \\
	II_{ijkl}=\delta_{ik}\delta_{jl} \text{,}\label{eq:Id4}
\end{gather}

\noindent onde $\delta_{ij}$ é a função delta de Kronecker, definida por

\begin{equation}\label{eq:kronecker}
	\delta_{ij}=\begin{cases}
		1\text{, se } i=j \\ 0\text{, se } i\ne j \text{.}
	\end{cases}
\end{equation}

\section{Derivadas tensoriais}

Nesta seção são enumeradas algumas derivadas de tensores recorrentes ao longo
do texto, onde $\mathbf{T}$ é um tensor genérico de segunda ordem.
\begin{itemize}[leftmargin=\parindent,labelwidth=\parindent,labelsep=0.3cm]
	\item Derivada do quadrado de um tensor:

	      \begin{equation}
		      \dfrac{\partial (\mathbf{T}^2)_{ij}}{\partial
			      {T}_{kl}}=\delta_{ik}{T}_{lj}+{T}_{ik}\delta_{jl}
		      \text{.} \label{eq:dT2}
	      \end{equation}

	\item Derivada da inversa de um tensor assimétrico:

	      \begin{equation}\label{eq:dinv}
		      \dfrac{\partial {T}^{-1}_{ij}}{\partial {T}_{kl}} =
		      -{T}^{-1}_{ik}{T}^{-1}_{lj}
		      \text{.}%=-\dfrac{1}{2}(\mathbf{T}^{-1}_{il} \mathbf{T}^{-1}_{kj}+\mathbf{T}^{-1}_{ik}\mathbf{T}^{-1}_{lj})   <= CASO SIMETRICO
	      \end{equation}

	\item Derivada da inversa de um tensor simétrico \footnote{Essa
		      expressão deve ser aplicada caso $\mathbf{T}$ seja
		      representado apenas pela sua
		      parte simétrica.}:

	      \begin{equation}\label{eq:dinvsim}
		      \dfrac{\partial {T}^{-1}_{ij}}{\partial {T}_{kl}} =
		      -\dfrac{1}{2}\left({T}^{-1}_{ik}{T}^{-1}_{jl} +
		      {T}^{-1}_{il}{T}^{-1}_{jk}\right)
		      \text{.}%=-\dfrac{1}{2}(\mathbf{T}^{-1}_{il} \mathbf{T}^{-1}_{kj}+\mathbf{T}^{-1}_{ik}\mathbf{T}^{-1}_{lj})   <= CASO SIMETRICO
	      \end{equation}

	\item Traço de um tensor:

	      \begin{equation}
		      \text{tr}(\mathbf{T})={T}_{ii} \text{,}
	      \end{equation}

	      \begin{equation}
		      \dfrac{\partial\, \text{tr}(\mathbf{T})}{\partial
			      \mathbf{T}}=\mathbf{I} \text{.} \label{eq:dtraco}
	      \end{equation}

	\item Tensor desviador:

	      \begin{equation}
		      \mathbf{T}^D = \mathbf{T} -
		      \dfrac{1}{3}\text{tr}(\mathbf{T})\mathbf{I}
		      \text{,} \label{eq:dev}
	      \end{equation}

	      \begin{equation}
		      \dfrac{\partial \mathbf{T}^D}{\partial \mathbf{T}} =
		      \mathbf{II} -
		      \dfrac{1}{3} \mathbf{I}\otimes\mathbf{I} \text{.}
		      \label{eq:ddev}
	      \end{equation}

	\item Derivada do determinante de um tensor:

	      \begin{equation}
		      \dfrac{\partial \det(\mathbf{T})}{\partial \mathbf{T}} =
		      \det(\mathbf{T})\cdot\mathbf{T}^{-T} \text{,}
		      \label{eq:ddet}
	      \end{equation}

	      \begin{equation}
		      \dfrac{\partial \sqrt{\det(\mathbf{T})}}{\partial
			      \mathbf{T}} =
		      \dfrac{1}{2}\sqrt{\det(\mathbf{T})}\cdot\mathbf{T}^{-T}
		      \text{.}
		      \label{eq:dsqrtdet}
	      \end{equation}

	\item Parcela simétrica de um tensor:

	      \begin{equation}
		      \text{sim}(\mathbf{T}) = \dfrac{1}{2}(\mathbf{T} +
		      \mathbf{T}^T)
		      \text{,} \label{eq:sim}
	      \end{equation}

	      \begin{equation}
		      \dfrac{\partial\, \text{sim}(\mathbf{T})}{\partial
			      \mathbf{T}} =
		      \dfrac{1}{2}(\mathbf{II} + \mathbf{II}^T)
		      \text{.}\label{eq:dsim}
	      \end{equation}

	\item Parcela antissimétrica de um tensor:

	      \begin{equation}
		      \text{ant}(\mathbf{T}) = \dfrac{1}{2}(\mathbf{T} -
		      \mathbf{T}^T)
		      \text{,} \label{eq:ant}
	      \end{equation}

	      \begin{equation}
		      \dfrac{\partial\, \text{ant}(\mathbf{T})}{\partial
			      \mathbf{T}} =
		      \dfrac{1}{2}(\mathbf{II} - \mathbf{II}^T)
		      \text{.}\label{eq:dant}
	      \end{equation}

	\item Norma de um tensor:

	      \begin{equation}\label{eq:dnorm}
		      \|\mathbf{T}\|=\sqrt{\mathbf{T}:\mathbf{T}} ,
	      \end{equation}

	      \begin{equation}
		      \dfrac{\partial\| \mathbf{T} \|}{\partial \mathbf{T}} =
		      \dfrac{\mathbf{T}}{\| \mathbf{T} \|} .
	      \end{equation}

\end{itemize}

%Onde a derivada apresentada em \eqref{eq:dinv} é válida apenas para tensores $\mathbf{T}$ simétricos.

\end{document}